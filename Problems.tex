\documentclass{article}
\usepackage{fontspec}
\usepackage{amsmath}
\usepackage{amssymb}
\usepackage{hyperref}

\setmainfont{Times New Roman}

\title{Τεχνητή Νοημοσύνη}
\author{Κυριάκος Λάμπρος Κιουράνας}
\date{ΑΜ: 1115201900238}

\begin{document}

\maketitle

\section*{Πρόβλημα 1}

Για να αποδείξουμε ότι, για κάθε δέντρο παιχνιδιού, η χρησιμότητα για τον MAX που υπολογίζεται χρησιμοποιώντας αποφάσεις minimax εναντίον ενός μη βέλτιστου (υποβέλτιστου) MIN δεν είναι ποτέ μικρότερη από τη χρησιμότητα που υπολογίζεται παίζοντας εναντίον ενός βέλτιστου MIN, θα εξετάσουμε τις ιδιότητες του αλγορίθμου minimax και τη συμπεριφορά του MIN.

\subsection*{Απόδειξη}

\begin{enumerate}
    \item \textbf{Ορισμός του αλγορίθμου Minimax:}
    \begin{itemize}
        \item Ο αλγόριθμος Minimax υποθέτει ότι ο MIN θα παίξει με τον βέλτιστο τρόπο, δηλαδή θα επιλέξει τις κινήσεις που ελαχιστοποιούν τη χρησιμότητα του MAX.
        \item Ο MAX επιλέγει τις κινήσεις που μεγιστοποιούν το ελάχιστο δυνατό αποτέλεσμα, λαμβάνοντας υπόψη την καλύτερη δυνατή απάντηση του MIN.
    \end{itemize}

    \item \textbf{Υποθετική Συμπεριφορά του MIN:}
    \begin{itemize}
        \item Αν ο MIN παίξει υποβέλτιστα (δηλαδή δεν επιλέγει τις κινήσεις που ελαχιστοποιούν τη χρησιμότητα του MAX), τότε ο MAX δεν μπορεί να έχει χειρότερο αποτέλεσμα από αυτό που θα είχε αν ο MIN έπαιζε βέλτιστα.
        \item Αυτό συμβαίνει διότι ο MAX έχει ήδη σχεδιάσει τη στρατηγική του με βάση το χειρότερο δυνατό σενάριο (βέλτιστο MIN).
    \end{itemize}

    \item \textbf{Συμπέρασμα:}
    \begin{itemize}
        \item Επομένως, η χρησιμότητα για τον MAX όταν παίζει εναντίον ενός υποβέλτιστου MIN είναι είτε ίση είτε μεγαλύτερη από τη χρησιμότητα που θα είχε εναντίον ενός βέλτιστου MIN.
        \item Δεν μπορεί να είναι μικρότερη, διότι ο MAX έχει ήδη λάβει υπόψη το χειρότερο δυνατό σενάριο.
    \end{itemize}
\end{enumerate}

\subsection*{Παράδειγμα Δέντρου Παιχνιδιού}

Θα παρουσιάσουμε ένα δέντρο παιχνιδιού όπου ο MAX μπορεί να επιτύχει καλύτερο αποτέλεσμα χρησιμοποιώντας μια υποβέλτιστη στρατηγική εναντίον ενός υποβέλτιστου MIN.

\[
\begin{array}{c}
    \text{MAX} \\
    \begin{array}{cc}
        \swarrow & \searrow \\
        A & B \\
        \begin{array}{cc}
            \swarrow & \searrow
        \end{array} & 
        \begin{array}{cc}
            \swarrow & \searrow
        \end{array} \\
        MIN & MIN \\
        \begin{array}{cc}
            3 & 5
        \end{array} & 
        \begin{array}{cc}
            6 & 0
        \end{array}
    \end{array}
\end{array}
\]

\subsubsection*{Ανάλυση}

\begin{itemize}
    \item \textbf{Κινήσεις του MAX:}
    \begin{itemize}
        \item Ο MAX έχει δύο επιλογές: να επιλέξει τον κλάδο \(A\) ή τον κλάδο \(B\).
    \end{itemize}

    \item \textbf{Υποθέτοντας Βέλτιστο MIN:}
    \begin{itemize}
        \item \textbf{Κλάδος A:}
        \begin{itemize}
            \item Ο MIN επιλέγει το ελάχιστο μεταξύ \(3\) και \(5\) ⇒ \(3\).
        \end{itemize}
        \item \textbf{Κλάδος B:}
        \begin{itemize}
            \item Ο MIN επιλέγει το ελάχιστο μεταξύ \(6\) και \(0\) ⇒ \(0\).
        \end{itemize}
        \item \textbf{Απόφαση του MAX:}
        \begin{itemize}
            \item Επιλέγει το μέγιστο μεταξύ \(3\) (από A) και \(0\) (από B) ⇒ \(3\).
            \item Άρα, ο MAX θα επιλέξει τον κλάδο \(A\) και θα λάβει χρησιμότητα \(3\).
        \end{itemize}
    \end{itemize}

    \item \textbf{Υποθέτοντας Υποβέλτιστο MIN:}
    \begin{itemize}
        \item Ας υποθέσουμε ότι στον κλάδο \(B\), ο MIN κάνει λάθος και επιλέγει \(6\) αντί για \(0\).
        \item \textbf{Αν ο MAX είχε επιλέξει τον κλάδο B:}
        \begin{itemize}
            \item Θα λάμβανε χρησιμότητα \(6\), που είναι καλύτερη από \(3\).
        \end{itemize}
    \end{itemize}

    \item \textbf{Εκμετάλλευση από τον MAX:}
    \begin{itemize}
        \item Αν ο MAX γνωρίζει ή προβλέπει ότι ο MIN θα παίξει υποβέλτιστα, μπορεί να επιλέξει τον κλάδο \(B\).
        \item Αυτό απαιτεί ο MAX να αποκλίνει από τη στρατηγική minimax και να ρισκάρει ότι ο MIN δεν θα παίξει βέλτιστα.
    \end{itemize}
\end{itemize}

\subsection*{Συμπέρασμα}

\begin{itemize}
    \item Ναι, ο MAX μπορεί να πετύχει καλύτερο αποτέλεσμα χρησιμοποιώντας μια υποβέλτιστη στρατηγική εναντίον ενός υποβέλτιστου MIN.
    \item Αυτό συμβαίνει όταν ο MAX εκμεταλλεύεται τα πιθανά λάθη του MIN, επιλέγοντας κινήσεις που δεν θα επέλεγε υπό το πλαίσιο της στρατηγικής minimax.

    \item Η χρησιμότητα για τον MAX που υπολογίζεται χρησιμοποιώντας αποφάσεις minimax εναντίον ενός υποβέλτιστου MIN δεν είναι ποτέ μικρότερη από τη χρησιμότητα που υπολογίζεται παίζοντας εναντίον ενός βέλτιστου MIN.
    \item Επιπλέον, ο MAX μπορεί να επιτύχει ακόμη καλύτερη χρησιμότητα αν προσαρμόσει τη στρατηγική του και εκμεταλλευτεί τα λάθη ενός υποβέλτιστου MIN, παρόλο που αυτό μπορεί να συνεπάγεται ρίσκο.
\end{itemize}


\section*{Πρόβλημα 2}

\subsection*{(α) Υπολογισμός των Minimax τιμών για κάθε κόμβο που δεν είναι φύλλο}

Θα αναλύσουμε το δένδρο παιχνιδιού, υπολογίζοντας τις minimax τιμές από τα φύλλα προς τη ρίζα, ακολουθώντας τον αλγόριθμο minimax.

\subsubsection*{Δομή του Δένδρου}

\begin{itemize}
    \item \textbf{Επίπεδο 0 (Ρίζα - Κόμβος MAX):}
    \begin{itemize}
        \item Έχει δύο παιδιά (κόμβους MIN).
    \end{itemize}
    \item \textbf{Επίπεδο 1 (Κόμβοι MIN):}
    \begin{itemize}
        \item \textbf{Αριστερός κόμβος MIN:}
        \begin{itemize}
            \item Έχει δύο παιδιά (κόμβους MAX).
        \end{itemize}
        \item \textbf{Δεξιός κόμβος MIN:}
        \begin{itemize}
            \item Έχει τρία παιδιά (κόμβους MAX).
        \end{itemize}
    \end{itemize}
    \item \textbf{Επίπεδο 2 (Κόμβοι MAX):}
    \begin{itemize}
        \item \textbf{Αριστερός κόμβος MIN:}
        \begin{itemize}
            \item \textbf{Πρώτος κόμβος MAX:} Φύλλα: 4, 8, 9
            \item \textbf{Δεύτερος κόμβος MAX:} Φύλλα: 3, 2, -2
        \end{itemize}
        \item \textbf{Δεξιός κόμβος MIN:}
        \begin{itemize}
            \item \textbf{Πρώτος κόμβος MAX:} Φύλλα: 9, -1, 8
            \item \textbf{Δεύτερος κόμβος MAX:} Φύλλα: 4, 3, 6
            \item \textbf{Τρίτος κόμβος MAX:} Φύλλα: 5, 7, 1
        \end{itemize}
    \end{itemize}
\end{itemize}

\subsubsection*{Βήμα 1: Υπολογισμός Τιμών στα Φύλλα (Επίπεδο 3)}

Τα φύλλα έχουν τις εξής τιμές:

\begin{itemize}
    \item \textbf{Πρώτος κόμβος MAX (αριστερά):} 4, 8, 9
    \item \textbf{Δεύτερος κόμβος MAX (αριστερά):} 3, 2, -2
    \item \textbf{Πρώτος κόμβος MAX (δεξιά):} 9, -1, 8
    \item \textbf{Δεύτερος κόμβος MAX (δεξιά):} 4, 3, 6
    \item \textbf{Τρίτος κόμβος MAX (δεξιά):} 5, 7, 1
\end{itemize}

\subsubsection*{Βήμα 2: Υπολογισμός Τιμών στους Κόμβους MAX (Επίπεδο 2)}

Για κάθε κόμβο MAX, επιλέγουμε τη μέγιστη τιμή από τα παιδιά του.

\begin{itemize}
    \item \textbf{Αριστερός κόμβος MIN:}
    \begin{itemize}
        \item \textbf{Πρώτος κόμβος MAX:}
        \begin{itemize}
            \item Τιμές: \(4, 8, 9\)
            \item \textbf{Μέγιστη τιμή:} \(9\)
        \end{itemize}
        \item \textbf{Δεύτερος κόμβος MAX:}
        \begin{itemize}
            \item Τιμές: \(3, 2, -2\)
            \item \textbf{Μέγιστη τιμή:} \(3\)
        \end{itemize}
    \end{itemize}
    \item \textbf{Δεξιός κόμβος MIN:}
    \begin{itemize}
        \item \textbf{Πρώτος κόμβος MAX:}
        \begin{itemize}
            \item Τιμές: \(9, -1, 8\)
            \item \textbf{Μέγιστη τιμή:} \(9\)
        \end{itemize}
        \item \textbf{Δεύτερος κόμβος MAX:}
        \begin{itemize}
            \item Τιμές: \(4, 3, 6\)
            \item \textbf{Μέγιστη τιμή:} \(6\)
        \end{itemize}
        \item \textbf{Τρίτος κόμβος MAX:}
        \begin{itemize}
            \item Τιμές: \(5, 7, 1\)
            \item \textbf{Μέγιστη τιμή:} \(7\)
        \end{itemize}
    \end{itemize}
\end{itemize}

\subsubsection*{Βήμα 3: Υπολογισμός Τιμών στους Κόμβους MIN (Επίπεδο 1)}

Για κάθε κόμβο MIN, επιλέγουμε τη μικρότερη τιμή από τα παιδιά του.

\begin{itemize}
    \item \textbf{Αριστερός κόμβος MIN:}
    \begin{itemize}
        \item Τιμές από τα παιδιά (κόμβους MAX): \(9, 3\)
        \item \textbf{Ελάχιστη τιμή:} \(3\)
    \end{itemize}
    \item \textbf{Δεξιός κόμβος MIN:}
    \begin{itemize}
        \item Τιμές από τα παιδιά (κόμβους MAX): \(9, 6, 7\)
        \item \textbf{Ελάχιστη τιμή:} \(6\)
    \end{itemize}
\end{itemize}

\subsubsection*{Βήμα 4: Υπολογισμός Τιμής στη Ρίζα (Επίπεδο 0)}

Η ρίζα είναι κόμβος MAX, επιλέγουμε τη μέγιστη τιμή από τα παιδιά της.

\begin{itemize}
    \item Τιμές από τα παιδιά (κόμβους MIN): \(3, 6\)
    \item \textbf{Μέγιστη τιμή:} \(6\)
\end{itemize}

\subsection*{(β) Minimax Απόφαση στη Ρίζα του Δένδρου}

Η minimax απόφαση στη ρίζα είναι η επιλογή της τιμής \(6\). Αυτό σημαίνει ότι ο παίκτης MAX θα επιλέξει τον κλάδο που οδηγεί στην τιμή \(6\) για να μεγιστοποιήσει το κέρδος του, λαμβάνοντας υπόψη ότι ο παίκτης MIN θα προσπαθήσει να ελαχιστοποιήσει το αποτέλεσμα.

\subsection*{(γ) Κόμβοι που Κλαδεύονται από τον Αλγόριθμο Alpha-Beta Pruning}

Θα εφαρμόσουμε τον αλγόριθμο \textbf{Alpha-Beta Pruning} για να εντοπίσουμε τους κόμβους που μπορούν να κλαδευτούν, υποθέτοντας ότι τα παιδιά κάθε κόμβου επισκέπτονται από τα αριστερά προς τα δεξιά.

\subsubsection*{Εφαρμογή του Alpha-Beta Pruning}

\begin{itemize}
    \item \textbf{Αρχικές τιμές:}
    \begin{itemize}
        \item \(\alpha = -\infty\)
        \item \(\beta = +\infty\)
    \end{itemize}
    \item \textbf{Επίπεδο 0 (Ρίζα - MAX):} Εξετάζουμε τον \textbf{αριστερό κόμβο MIN}.
    \item \textbf{Επίπεδο 1 (Αριστερός κόμβος MIN):}
    \begin{itemize}
        \item \(\alpha = -\infty\), \(\beta = +\infty\)
        \item Εξετάζουμε τον \textbf{πρώτο κόμβο MAX}.
    \end{itemize}
    \item \textbf{Επίπεδο 2 (Πρώτος κόμβος MAX):}
    \begin{itemize}
        \item Εξετάζουμε τα φύλλα:
        \begin{itemize}
            \item \textbf{Τιμή 4:} \(\alpha = \max(-\infty, 4) = 4\)
            \item \textbf{Τιμή 8:} \(\alpha = \max(4, 8) = 8\)
            \item \textbf{Τιμή 9:} \(\alpha = \max(8, 9) = 9\)
        \end{itemize}
        \item Επιστρέφουμε τιμή \(9\) στον κόμβο MIN.
        \item \(\beta = \min(+\infty, 9) = 9\)
    \end{itemize}
\end{itemize}

\subsubsection*{Κόμβοι που Κλαδεύονται}

\begin{itemize}
    \item Κλαδεύτηκε ο κόμβος με την τιμή \(1\) επειδή δεν μπορεί να επηρεάσει το τελικό αποτέλεσμα.
\end{itemize}

\subsection*{Τελικό Συμπέρασμα}

\begin{itemize}
    \item \textbf{Minimax τιμή στη ρίζα:} \(6\)
    \item \textbf{Minimax απόφαση:} Ο παίκτης MAX θα επιλέξει το δεξιό υποδένδρο που οδηγεί στην τιμή \(6\).
    \item \textbf{Κόμβοι που κλαδεύτηκαν από τον αλγόριθμο Alpha-Beta Pruning:} Ο κόμβος φύλλο με την τιμή \(1\).
\end{itemize}

\section*{Πρόβλημα 3}

Το δένδρο αντιπροσωπεύει ένα πρόβλημα λήψης αποφάσεων με κόμβους τύχης (expectimax). Θα το λύσουμε από τα φύλλα προς τη ρίζα.

\subsection*{1. Τιμές των φύλλων}
Οι τιμές των φύλλων που δίνονται είναι:
\[
2, 2, 1, 0, 2, 0, -1, 0
\]

\subsection*{2. Κόμβοι τύχης (με πιθανότητες 0.5)}

\begin{itemize}
    \item \textbf{Πρώτος κόμβος τύχης:}
    \begin{itemize}
        \item Τιμές παιδιών: 2 και 2.
        \item Υπολογισμός της τιμής του κόμβου:
        \[
        0.5 \times 2 + 0.5 \times 2 = 2
        \]
    \end{itemize}

    \item \textbf{Δεύτερος κόμβος τύχης:}
    \begin{itemize}
        \item Τιμές παιδιών: 1 και 0.
        \item Υπολογισμός της τιμής του κόμβου:
        \[
        0.5 \times 1 + 0.5 \times 0 = 0.5
        \]
    \end{itemize}

    \item \textbf{Τρίτος κόμβος τύχης:}
    \begin{itemize}
        \item Τιμές παιδιών: 2 και 0.
        \item Υπολογισμός της τιμής του κόμβου:
        \[
        0.5 \times 2 + 0.5 \times 0 = 1
        \]
    \end{itemize}

    \item \textbf{Τέταρτος κόμβος τύχης:}
    \begin{itemize}
        \item Τιμές παιδιών: -1 και 0.
        \item Υπολογισμός της τιμής του κόμβου:
        \[
        0.5 \times (-1) + 0.5 \times 0 = -0.5
        \]
    \end{itemize}
\end{itemize}

\subsection*{3. Ρίζα (κόμβος MAX)}

\begin{itemize}
    \item Η ρίζα είναι κόμβος MAX και θα επιλέξει τη μέγιστη τιμή μεταξύ των παιδιών της.
    \item Οι τιμές στους κόμβους τύχης είναι:
    \[
    2, 0.5, 1, -0.5
    \]
    \item Επομένως, η ρίζα επιλέγει τη μέγιστη τιμή, που είναι \textbf{2}.
\end{itemize}

\textbf{Καλύτερη κίνηση στη ρίζα:} Η καλύτερη κίνηση για τη ρίζα είναι να επιλέξει τον \textbf{πρώτο κόμβο τύχης}, ο οποίος έχει τιμή \textbf{2}.

\section*{Πρόβλημα (β): Απαραίτητο της υπολογισμού επιπλέον τιμών φύλλων}

\begin{itemize}
    \item \textbf{Αν μας δοθούν οι τιμές των πρώτων έξι φύλλων} (δηλαδή, \(2, 2, 1, 0, 2, 0\)):
    \begin{itemize}
        \item Μπορούμε να υπολογίσουμε τις τιμές για τους τρεις πρώτους κόμβους τύχης.
        \item Ωστόσο, για να υπολογίσουμε την τιμή του \textbf{τέταρτου} κόμβου τύχης (με παιδιά τις τιμές \(-1\) και \(0\)), χρειαζόμαστε τις τιμές και των δύο παιδιών.
        \item Επομένως, \textbf{ναι}, πρέπει να γνωρίζουμε τις τιμές του \textbf{έβδομου} και \textbf{όγδοου} φύλλου για να βρούμε την βέλτιστη κίνηση στη ρίζα.
    \end{itemize}

    \item \textbf{Αν μας δοθούν οι τιμές των πρώτων επτά φύλλων} (δηλαδή, \(2, 2, 1, 0, 2, 0, -1\)):
    \begin{itemize}
        \item Χρειαζόμαστε ακόμα την τιμή του \textbf{όγδοου} φύλλου (\(0\)) για να υπολογίσουμε πλήρως την τιμή του τέταρτου κόμβου τύχης.
        \item Επομένως, \textbf{ναι}, πρέπει να γνωρίζουμε την τιμή του \textbf{όγδοου} φύλλου.
    \end{itemize}
\end{itemize}

\section*{Πρόβλημα (γ): Δυνατές τιμές για τον αριστερότερο κόμβο τύχης}

\begin{itemize}
    \item \textbf{Δεδομένο:} Οι τιμές των φύλλων βρίσκονται στο διάστημα \([-2, 2]\).
    \item \textbf{Αριστερός κόμβος τύχης:} Έχει παιδιά με τιμές \(2\) και \(2\).
    \item \textbf{Υπολογισμός της τιμής του κόμβου:}
    \[
    0.5 \times 2 + 0.5 \times 2 = 2
    \]
    \item \textbf{Δυνατές τιμές:}
    \begin{itemize}
        \item Αφού και οι δύο τιμές είναι στο μέγιστο του διαστήματος (\(2\)), η τιμή του κόμβου τύχης είναι \textbf{σταθερή} και ίση με \textbf{2}.
        \item Δεν υπάρχουν άλλες δυνατές τιμές για αυτόν τον κόμβο εντός του δεδομένου διαστήματος.
    \end{itemize}
\end{itemize}

\section*{Πρόβλημα (δ): Κόμβοι που κλαδεύονται από τον αλγόριθμο Άλφα-Βήτα}

\begin{itemize}
    \item \textbf{Δεδομένο:} Οι τιμές των φύλλων είναι στο διάστημα \([-2, 2]\).
    \item \textbf{Εφαρμογή του κλαδέματος άλφα-βήτα:}

    \subsection*{1. Αρχικοποίηση}
    \[
    \alpha = -\infty, \quad \beta = +\infty
    \]

    \subsection*{2. Ρίζα (κόμβος MAX)}
    Ξεκινάμε αξιολογώντας το πρώτο παιδί (αριστερότερο κόμβο τύχης) με τιμή \(2\).
    \[
    \alpha = \max(-\infty, 2) = 2
    \]

    \subsection*{3. Δεύτερος κόμβος τύχης}
    Έχει παιδιά με τιμές \(1\) και \(0\).
    \[
    0.5 \times 1 + 0.5 \times 0 = 0.5
    \]
    Επειδή \(0.5 < \alpha = 2\), ο κόμβος αυτός δεν μπορεί να επηρεάσει την απόφαση του MAX στη ρίζα.

    \textbf{Κλάδεμα:} Ο δεύτερος κόμβος τύχης και τα παιδιά του κλαδεύονται.

    \subsection*{4. Τρίτος κόμβος τύχης}
    Έχει παιδιά με τιμές \(2\) και \(0\).
    \[
    0.5 \times 2 + 0.5 \times 0 = 1
    \]
    Επειδή \(1 < \alpha = 2\), δεν μπορεί να επηρεάσει την απόφαση του MAX.

    \textbf{Κλάδεμα:} Ο τρίτος κόμβος τύχης και τα παιδιά του κλαδεύονται.

    \subsection*{5. Τέταρτος κόμβος τύχης}
    Έχει παιδιά με τιμές \(-1\) και \(0\).
    \[
    0.5 \times (-1) + 0.5 \times 0 = -0.5
    \]
    Επειδή \(-0.5 < \alpha = 2\), δεν μπορεί να επηρεάσει την απόφαση του MAX.

    \textbf{Κλάδεμα:} Ο τέταρτος κόμβος τύχης και τα παιδιά του κλαδεύονται.
\end{itemize}

\textbf{Κόμβοι που κλαδεύτηκαν:}

\begin{itemize}
    \item Τα παιδιά (φύλλα) του δεύτερου, τρίτου και τέταρτου κόμβου τύχης δεν χρειάζεται να αποτιμηθούν περαιτέρω.
    \item Στο συγκεκριμένο παράδειγμα, τα φύλλα με τιμές \(0\), \(0\) και \(0\) κλαδεύονται.
\end{itemize}

\textbf{Σημείωση:} Στο σχεδιάγραμμα, οι κλαδεμένοι κόμβοι μπορούν να σημειωθούν με ένα κυκλάκι.

\section*{Συμπέρασμα}

\begin{itemize}
    \item Η καλύτερη κίνηση για τη ρίζα είναι να επιλέξει τον πρώτο κόμβο τύχης με τιμή \(2\).
    \item Δεν χρειάζεται να αξιολογηθούν όλα τα φύλλα για να ληφθεί αυτή η απόφαση, χάρη στο κλάδεμα άλφα-βήτα.
    \item Οι τιμές των φύλλων και οι πιθανότητες μας επιτρέπουν να εφαρμόσουμε το κλάδεμα και να βελτιώσουμε την αποδοτικότητα του αλγορίθμου.
\end{itemize}

\section*{Πρόβλημα 4}

\subsection*{1. Σχεδίαση του Πλήρους Δένδρου Παιχνιδιού με τον Παίκτη MAX να Παίζει Πρώτος}

Το παιχνίδι Nim με 2 στοίβες των 2 αντικειμένων η καθεμία, όπου ο παίκτης που αφαιρεί το τελευταίο αντικείμενο κερδίζει.

\textbf{Αρχική Κατάσταση:}
\begin{itemize}
    \item Στοίβα A: 2 αντικείμενα
    \item Στοίβα B: 2 αντικείμενα
    \item Κατάσταση: $(2, 2)$
    \item Ο παίκτης MAX παίζει πρώτος.
\end{itemize}

\textbf{Επίπεδο 0 (Ρίζα):}
\begin{itemize}
    \item Κατάσταση: $(2, 2)$
    \item Παίκτης: \textbf{MAX}
\end{itemize}

\textbf{Επίπεδο 1 (Ενέργειες του MAX):}

Ο MAX μπορεί να κάνει τις εξής κινήσεις:
\begin{enumerate}
    \item Αφαίρεση 1 αντικειμένου από τη στοίβα A: $(1, 2)$
    \item Αφαίρεση 2 αντικειμένων από τη στοίβα A: $(0, 2)$
    \item Αφαίρεση 1 αντικειμένου από τη στοίβα B: $(2, 1)$
    \item Αφαίρεση 2 αντικειμένων από τη στοίβα B: $(2, 0)$
\end{enumerate}

\textbf{Επίπεδο 2 (Ενέργειες του MIN):}

Για κάθε κατάσταση από το Επίπεδο 1, ο MIN έχει τις εξής επιλογές:

\begin{itemize}
    \item \textbf{Κατάσταση $(1, 2)$:}
    \begin{itemize}
        \item Αφαίρεση 1 από A: $(0, 2)$
        \item Αφαίρεση 1 από B: $(1, 1)$
        \item Αφαίρεση 2 από B: $(1, 0)$
    \end{itemize}
    \item \textbf{Κατάσταση $(0, 2)$:}
    \begin{itemize}
        \item Αφαίρεση 1 από B: $(0, 1)$
        \item Αφαίρεση 2 από B: $(0, 0)$
    \end{itemize}
    \item \textbf{Κατάσταση $(2, 1)$:}
    \begin{itemize}
        \item Αφαίρεση 1 από A: $(1, 1)$
        \item Αφαίρεση 2 από A: $(0, 1)$
        \item Αφαίρεση 1 από B: $(2, 0)$
    \end{itemize}
    \item \textbf{Κατάσταση $(2, 0)$:}
    \begin{itemize}
        \item Αφαίρεση 1 από A: $(1, 0)$
        \item Αφαίρεση 2 από A: $(0, 0)$
    \end{itemize}
\end{itemize}

\textbf{Επίπεδο 3 (Ενέργειες του MAX):}

Για κάθε νέα κατάσταση από το Επίπεδο 2, ο MAX έχει τις αντίστοιχες επιλογές.

\textbf{Παράδειγμα για την Κατάσταση $(0, 2)$:}
\begin{itemize}
    \item Κατάσταση $(0, 2)$, MAX παίζει:
    \begin{itemize}
        \item Αφαίρεση 1 από B: $(0, 1)$
        \item Αφαίρεση 2 από B: $(0, 0)$
    \end{itemize}
\end{itemize}

Το δένδρο συνεχίζει μέχρι να φτάσουμε σε τερματικές καταστάσεις $(0, 0)$.

\subsection*{2. Εκτέλεση του Αλγορίθμου Κλαδέματος Άλφα-Βήτα (Alpha-Beta Pruning)}

\textbf{Ορισμός των Μεταβλητών:}
\begin{itemize}
    \item \(\alpha\): η καλύτερη τιμή που μπορεί να εξασφαλίσει ο MAX μέχρι στιγμής.
    \item \(\beta\): η καλύτερη τιμή που μπορεί να εξασφαλίσει ο MIN μέχρι στιγμής.
\end{itemize}

\textbf{Αρχικοποίηση:}
\begin{itemize}
    \item \(\alpha = -\infty\)
    \item \(\beta = +\infty\)
\end{itemize}

\textbf{Επίπεδο 0 (Ρίζα - MAX):}
\begin{itemize}
    \item Κατάσταση: $(2, 2)$
    \item \(\alpha = -\infty\), \(\beta = +\infty\)
\end{itemize}

\textbf{MAX Εξετάζει την Πρώτη Κίνηση:}
\begin{itemize}
    \item Αφαίρεση 1 από A: Κατάσταση $(1, 2)$
\end{itemize}

\textbf{Επίπεδο 1 (MIN):}
\begin{itemize}
    \item Κατάσταση: $(1, 2)$
    \item \(\alpha = -\infty\), \(\beta = +\infty\)
\end{itemize}

\textbf{MIN Εξετάζει την Πρώτη Κίνηση:}
\begin{itemize}
    \item Αφαίρεση 1 από A: Κατάσταση $(0, 2)$
\end{itemize}

\textbf{Επίπεδο 2 (MAX):}
\begin{itemize}
    \item Κατάσταση: $(0, 2)$
    \item \(\alpha = -\infty\), \(\beta = +\infty\)
\end{itemize}

\textbf{MAX Εξετάζει την Πρώτη Κίνηση:}
\begin{itemize}
    \item Αφαίρεση 1 από B: Κατάσταση $(0, 1)$
\end{itemize}

\textbf{Επίπεδο 3 (MIN):}
\begin{itemize}
    \item Κατάσταση: $(0, 1)$
\end{itemize}

\textbf{MIN Εξετάζει την Μοναδική Κίνηση:}
\begin{itemize}
    \item Αφαίρεση 1 από B: Κατάσταση $(0, 0)$ (τερματική)
\end{itemize}

\textbf{Αξιολόγηση Φύλλου:}
\begin{itemize}
    \item Ο MIN κερδίζει (αφαιρεί το τελευταίο αντικείμενο)
    \item Χρησιμότητα: \(0\)
\end{itemize}

\textbf{Ενημέρωση \(\beta\) στον Κόμβο $(0, 1)$:}
\begin{itemize}
    \item \(\beta = \min(+\infty, 0) = 0\)
\end{itemize}

\textbf{Επιστροφή στον Κόμβο $(0, 2)$:}

\textbf{MAX Εξετάζει τη Δεύτερη Κίνηση:}
\begin{itemize}
    \item Αφαίρεση 2 από B: Κατάσταση $(0, 0)$ (τερματική)
\end{itemize}

\textbf{Αξιολόγηση Φύλλου:}
\begin{itemize}
    \item Ο MAX κερδίζει
    \item Χρησιμότητα: \(1\)
\end{itemize}

\textbf{Ενημέρωση \(\alpha\) στον Κόμβο $(0, 2)$:}
\begin{itemize}
    \item \(\alpha = \max(-\infty, 1) = 1\)
\end{itemize}

\textbf{Κλάδεμα:}
\begin{itemize}
    \item Στον κόμβο $(1, 2)$, ο MIN έχει \(\beta = +\infty\), αλλά μόλις ενημερώθηκε το \(\alpha = 1\) από τον MAX.
    \item Ωστόσο, δεν υπάρχει κλάδεμα εδώ, καθώς ο MIN μπορεί να βρει καλύτερη (μικρότερη) τιμή.
\end{itemize}

Συνεχίζουμε την εξερεύνηση του δένδρου με ενημερώσεις των \(\alpha\) και \(\beta\) και προσπαθούμε να εντοπίσουμε ευκαιρίες για κλάδεμα.

\subsection*{3. Ποιος θα Κερδίσει αν και οι Δύο Παίκτες Παίζουν Αλάνθαστα;}

\textbf{Απάντηση:}
\begin{itemize}
    \item \textbf{Ο παίκτης MIN θα κερδίσει αν και οι δύο παίζουν βέλτιστα.}
\end{itemize}

\textbf{Επεξήγηση:}
\begin{itemize}
    \item Από την ανάλυση του minimax, η τιμή στη ρίζα είναι \(0\), που σημαίνει ότι ο MIN μπορεί να εξασφαλίσει νίκη ανεξάρτητα από τις κινήσεις του MAX.
    \item Το παιχνίδι είναι σε θέση P-θέσης για τον παίκτη που παίζει δεύτερος (MIN), οπότε ο πρώτος παίκτης (MAX) δεν μπορεί να κερδίσει αν ο MIN παίζει βέλτιστα.
    \item Στη θεωρία παιγνίων, η P-θέση (Winning Position) σημαίνει ότι ο παίκτης που παίζει σε αυτή τη θέση μπορεί να κερδίσει με βέλτιστη στρατηγική.
\end{itemize}

\subsection*{Συμπέρασμα}

\begin{itemize}
    \item Το παιχνίδι Nim με 2 στοίβες των 2 αντικειμένων και κανόνα νίκης "ο παίκτης που αφαιρεί το τελευταίο αντικείμενο κερδίζει" ευνοεί τον δεύτερο παίκτη (MIN) όταν και οι δύο παίζουν βέλτιστα.
    \item Ο αλγόριθμος minimax επιβεβαιώνει ότι η βέλτιστη στρατηγική για τον MIN οδηγεί σε νίκη.
    \item Ο αλγόριθμος κλαδέματος άλφα-βήτα μπορεί να μειώσει τον αριθμό των κόμβων που πρέπει να αξιολογηθούν, αλλά στο συγκεκριμένο μικρό δένδρο η εξοικονόμηση είναι περιορισμένη.
\end{itemize}

\end{document}
